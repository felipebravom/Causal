%\documentclass[mathserif]{beamer}
\documentclass[handout]{beamer}
%\usetheme{Goettingen}
\usetheme{Warsaw}
%\usetheme{Singapore}
%\usetheme{Frankfurt}
%\usetheme{Copenhagen}
%\usetheme{Szeged}
%\usetheme{Montpellier}
%\usetheme{CambridgeUS}
%\usecolortheme{}
%\setbeamercovered{transparent}
\usepackage[english, activeacute]{babel}
\usepackage[utf8]{inputenc}
\usepackage{amsmath, amssymb}
\usepackage{dsfont}
\usepackage{graphics}
\usepackage{cases}
\usepackage{graphicx}
\usepackage{pgf}
\usepackage{epsfig}
\usepackage{amssymb}
\usepackage{multirow}	
\usepackage{amstext}
\usepackage[ruled,vlined,lined]{algorithm2e}
\usepackage{amsmath}
\usepackage{epic}
\usepackage{epsfig}
\usepackage{fontenc}
\usepackage{framed,color}
\usepackage{palatino, url, multicol}
\usepackage{listings}
%\algsetup{indent=2em}


\vspace{-0.5cm}
\title{Introduction to Causal Inference}
\vspace{-0.5cm}
\author[Felipe Bravo Márquez]{\footnotesize
%\author{\footnotesize  
 \textcolor[rgb]{0.00,0.00,1.00}{Felipe José Bravo Márquez}} 
\date{ \today }




\begin{document}
\begin{frame}
\titlepage


\end{frame}


%%%%%%%%%%%%%%%%%%%%%%%%%%%


\begin{frame}{Motivation}
\scriptsize{
\begin{itemize}
\item Our starting point is the difference between an observation and an intervention (or action). 
\item We can answer many questions from passive observation alone.
\item For example: do 16 year-old drivers have a higher incidence rate of traffic accidents than 18 year-old drivers? 
\item The answer corresponds to a difference of conditional probabilities.

\item Let random variables $I,A$ correspond to traffic incident rate and driver's age correspondingly:
\begin{displaymath}
 P(I|A=16)-P(I|A=18)>0?
\end{displaymath}


\item Both conditional probabilities can be estimated from a large enough sample drawn from the distribution.

\item The answer to the question we asked is solidly in the realm of observational statistics.

\item However, important questions often are not observational in nature.

\end{itemize}

\footnotemark{These slides are mainly based on Chapter 9 of \cite{hardt2021patterns}.}

} 

\end{frame}




%%%%%%%%%%%%%%%%%%%%%%%%%%%
\begin{frame}[allowframebreaks]\scriptsize
\frametitle{References}
\bibliography{bio}
\bibliographystyle{apalike}
%\bibliographystyle{flexbib}
\end{frame}  









%%%%%%%%%%%%%%%%%%%%%%%%%%%

\end{document}
